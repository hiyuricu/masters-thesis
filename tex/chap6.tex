\chapter{おわりに}\chaplab{conclusion}
本研究ではユーザがタイピングゲームを行うことでスペリング誤りを抽出する手法を提案した.ユーザ7名にタイピングゲームのプレイを依頼し,4,724回の英単語タイピングログを収集し,859個のスペリング誤りを抽出した.抽出されたスペリング誤りに対して分析を行うことで,タイピングゲームのような通常より素早くタイピングを行ったり,文字を書き写すような状況では,キー配置によって起こる誤りや入力すべき文字を飛ばしてしまう誤りがよく起きることがわかった.

本研究において,

\begin{itemize}
 \item 本研究のタイピングゲームでは速くタイピングゲームをクリアするほどより高いスコアを獲得することができ,ユーザは急いで文字を入力するため,打鍵ミスによるスペリング誤りの割合が大きくなる.
 \item ユーザは入力する文字を見て入力しているため,音韻的混同が原因によるスペリング誤りの頻度が相対的に小さくなる.
 \item Babaらが抽出した最終出力文字列に残らない,ユーザが文字を入力中に修正するスペリング誤り\cite{babaACL2012}と最終出力文字列に残る一般的なスペリング誤りの両方を抽出するため,それぞれの誤りが混合した結果になる.
\end{itemize}

\noindent
という仮説を立てスペリング誤りに関する定量的な分析を行ったが,その仮説が正しいということがわかった.
また我々が抽出したスペリング誤りの結果とBabaらが抽出したスペリング誤りの結果を比較して,それぞれの結果が似たような結果であることがわかったため,タイピングゲームを利用して抽出したスペリング誤りがスペリング誤り訂正に貢献するために活用できると考えられる.

しかし本研究での実験設定ではユーザが入力しようとしている英単語はわかっているが,文字の挿入誤りと置換誤りの区別ができない.
区別を可能にするためには,
\begin{comment}
タイピングゲームにおいて入力した文字列が正しいかどうか判定するときにエンターキーを入力させて英単語に対応する文字列を明確にする,
\end{comment}
Babaらの研究\cite{babaACL2012}のようにバックスペースを入力させて文字を消去させる必要があると考えられる.
また本研究での実験設定ではわからないが,ユーザによってスペリング誤りの頻度や傾向が変わっていることが考えられる.そのためタイピングゲームを行うときにユーザにアカウントを登録させてユーザそれぞれを区別できるようにすることで,ユーザごとの誤り訂正モデルの構築が可能であると考えられる.

\begin{comment}
タイピングゲームにおいて1文字入力するときの入力時間を記録しておくことでユーザがスペリング誤りを起こしていることを気付いて文字を打ち止めるときの文字と文字の間の境界をわかるようにするようにしたり,Babaらの研究\cite{babaACL2012}のようにバックスペースの入力をさせるようにしてスペリング誤りが修正されるものかそうでないかを明確にする必要があると考えられる.
\end{comment}

\begin{comment}
有用なログを抽出するために英単語以外にも英語の文章や日本語の単語,文章に対応するようなゲームの設計を行いたい.
\end{comment}

今後はユーザに対して教育的であるようなデータセットの利用,実験設定を検討したい.
たとえば英語のリスニングゲームとしてタイピングゲームを発展させることで,新たなスペリング誤りの傾向の発見や研究において有用なログの作成,ユーザにとって有益となるようなコンテンツの構築に繋げていきたい.またタイピングゲームをアプリケーションに繋げることで大規模なスペリング誤りの抽出を可能にしたい.

\newpage

\section*{\Large 謝辞}
\addcontentsline{toc}{chapter}{謝辞}
本論文の作成や研究生活など終始適切な助言を賜り,ご指導頂いた指導教員の小町守准教授に感謝致します.

本論文の副査を快く引き受けて下さった石川博教授,高間康史教授に感謝致します.

小町研究室のメンバーとは研究活動において日常的に様々な議論をして多くの知識や示唆を頂きました.また研究以外の活動を共に行うことで精神的にも支えられました.ありがとうございました.

本実験を行うにあたりタイピングゲームをプレイして頂いたユーザ7名に対して深く感謝いたします.

