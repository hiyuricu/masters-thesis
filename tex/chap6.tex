\chapter{おわりに}\chaplab{conclusion}
本研究ではユーザがタイピングゲームを行うことでスペリング誤りを抽出する手法を提案した.ユーザ7名にタイピングゲームのプレイを依頼し,4,724回の英単語タイピングログを収集し,859個のスペリング誤りを抽出した.抽出されたスペリング誤りに対して分析を行うことで,タイピングゲームのような通常より素早くタイピングを行ったり,文字を書き写すような状況では,キー配置によって起こる誤りや入力すべき文字を飛ばしてしまう誤りがよく起きることがわかった.

本研究において,

\begin{itemize}
 \item タイピングゲームにおいてユーザはハイスコアを獲得するために急いで文字を入力するため,打鍵ミスによるスペリング誤りの割合が大きくなる.
 \item ユーザは入力する文字を見て入力しているため,音韻的混同が原因によるスペリング誤りの頻度が相対的に小さくなる.
 \item 荒巻らが抽出したスペリング誤り\cite{aramakiNLP2010}とBabaらが抽出したスペリング誤り\cite{babaACL2012}の両方を抽出するため,それぞれ結果が混合した結果になる.
\end{itemize}

を仮定し,スペリング誤りに関する定量的な分析を行ったが,その結果が概ね正しいということがわかった.
また本研究のスペリング誤りの脱落誤りの割合の結果がBabaらが抽出したスペリング誤りの脱落誤りの割合の結果\cite{babaACL2012}と近い結果になったため,スペリング誤りを抽出するためにタイピングゲームを利用することに意義があると考えられる.

しかし本研究での実験設定ではユーザが入力しようとしている英単語はわかっているが,文字の挿入誤りと置換誤りの区別ができない.
区別を可能にするためには,
\begin{comment}
タイピングゲームにおいて入力した文字列が正しいかどうか判定するときにエンターキーを入力させて英単語に対応する文字列を明確にする,
\end{comment}
Babaらの研究\cite{babaACL2012}のようにバックスペースを入力させて文字を消去させる必要があると考えられる.
また本研究での実験設定ではわからないが,ユーザによってスペリング誤りの頻度や傾向が変わっていることが考えられる.そのためタイピングゲームを行うときにユーザにアカウントを登録させてユーザそれぞれを区別できるようにすることで,ユーザごとの誤り訂正モデルの構築が可能であると考えられる.

\begin{comment}
タイピングゲームにおいて1文字入力するときの入力時間を記録しておくことでユーザがスペリング誤りを起こしていることを気付いて文字を打ち止めるときの文字と文字の間の境界をわかるようにするようにしたり,Babaらの研究\cite{babaACL2012}のようにバックスペースの入力をさせるようにしてスペリング誤りが修正されるものかそうでないかを明確にする必要があると考えられる.
\end{comment}

\begin{comment}
有用なログを抽出するために英単語以外にも英語の文章や日本語の単語,文章に対応するようなゲームの設計を行いたい.
\end{comment}

今後はユーザに対して教育的であるようなデータセットの利用,実験設定を検討したい.
たとえば英語のリスニングゲームとしてタイピングゲームを発展させることで,新たなスペリング誤りの傾向の発見や研究において有用なログの作成,ユーザにとって有益となるようなコンテンツの構築に繋げていきたい.

\newpage

\section*{\Large 謝辞}
\addcontentsline{toc}{chapter}{謝辞}
実験を行うにあたりタイピングゲームをプレイして頂いたユーザ7名に対して深く感謝いたします.