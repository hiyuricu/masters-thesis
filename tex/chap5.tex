\chapter{スペリング誤りの結果と分析}
情報工学系の日本人大学院生7人にタイピングゲームを行わせた.ユーザが英単語を入力した回数が合計で4,724回あり,そのなかでユーザが英単語をタイプするとき1度はスペリング誤りを起こしている場合が712回存在した.アルファベットの文字列と英単語の編集距離は最大で10の差が存在した.ユーザが入力した1つの文字列には複数のスペリング誤りが存在する場合があり,そういった場合を考慮するとスペリング誤りの合計は859個存在した.それらのスペリング誤りに対して分析を行う.また情報工学系の日本人大学院生7人のそれぞれのユーザにおいて実験結果の個人差があることが考えられるが,そのことに関する比較実験は行っていない.
スペリング誤りの分析では定量的な分析と定性的な分析を行う.定量的な分析においては,タイピングゲームを利用する状況での仮定を正しいかどうかを考察する.

{\tabcolsep = 0.4mm
 \begin{table*}[t]
  \small
  \begin{center}
   \caption{スペリング誤りの文字と頻度(入力すべき文字が語頭または語末の場合は\_を示す)}
   \begin{tabular}{|c|c|c|c|c|c|c|c|} \hline
       	スペリング誤りの文字 & 頻度 & 1つ前の文字 & 頻度 & 入力すべき文字 & 頻度 & 1つ後の文字 & 頻度\\ \hline
	    e & 93 & r & 14 & r & 17 & e & 36\\ \hline
	    r & 73 & e & 12 & e & 19 & r & 26\\ \hline
	    s & 67 & \_ & 21 & c & 17 & e & 15\\ \hline
	    i & 60 & \_ & 9 & o & 18 & i & 30\\ \hline
	    o & 59 & \_ & 19 & p & 12 & o & 24\\ \hline
	    n & 56 & i & 24 & o & 22 & n & 40\\ \hline
	    a & 54 & \_ & 14 & e & 10 & a & 22\\ \hline
	    t & 54 & g & 9 & h, r & 8 & t & 21\\ \hline
	    u & 35 & o & 8 & y & 11 & u & 14\\ \hline
	    d & 34 & \_ & 7 & s & 10 & \_ & 8\\ \hline
	    g & 34 & n & 11 & t & 8 & \_ & 17\\ \hline
	    l & 31 & m & 6 & p & 9 & l & 21\\ \hline
	    h & 25 & \_ & 7 & f & 5 & h, o & 5\\ \hline
	    v & 25 & \_ & 9 & b & 12 & e & 11\\ \hline
	    p & 24 & l, t & 5 & o & 16 & p & 5\\ \hline
	    y & 23 & a & 4 & t & 9 & y & 9\\ \hline
	    c & 20 & e, i & 4 & v & 6 & c, e & 6\\ \hline
	    k & 20 & a & 8 & l & 14 & \_ & 9\\ \hline
	    w & 20 & \_ & 8 & e & 14 & w & 5\\ \hline
	    b & 15 & i & 4 & v & 9 & e & 9\\ \hline
	    m & 14 & \_ & 4 & n & 4 & m & 3\\ \hline
	    f & 11 & \_ & 5 & r & 3 & \_ & 3\\ \hline
	    x & 3 & e, n, o & 1 & c & 2 & e & 2\\ \hline
	    j & 3 & c & 2 & k & 2 & \_ & 2\\ \hline
	    z & 2 & e, \_ & 1 & a, v & 1 & i, m & 1\\ \hline
	    q & 1 & p & 1 & e & 1 & c & 1\\ \hline
   \end{tabular}
  \end{center}
 \end{table*}
}
 
\section{誤りに関する定量的な分析}
抽出したスペリング誤りの文字を頻度順に並べ,そのスペリング誤りの文字に対して,入力すべき文字と入力すべき文字の前後の文字の最も頻度の高かった文字を表5.1に示す.頻度が等しい場合は複数の文字を示してある.
入力すべき文字が文字列に対応した英単語の語頭や語末の文字であった場合,表5.1には記号の\_を表示している.
またそれぞれのアルファベットの誤りに対して,全ての入力すべき文字と入力すべき文字の前後の文字の頻度を付録Aにおいて示す.
表5.1の結果からa, i, u, e, oといった母音がスペリング誤りの文字の頻度の上位10件に含まれており,これは単語を入力する上で母音を入力する頻度が多いからだと考えられる.またタイピングゲームにおいて入力すべき文字の1つ後の文字を間違えて入力してしまう場合の割合が34.2\%,入力すべき文字の1つ前の文字を間違えて入力してしまう場合の割合が4.9\%となっており,入力すべき文字を繰り返して入力してしまうことよりも,入力すべき文字を1文字飛ばして入力している場合がよく起きていることがわかる.\footnote{本研究では文字の挿入誤りと置換誤りの区別ができない}これはタイピングゲームではハイスコアを競って急いで文字を入力しようとすることが原因だと考えられる.
またBabaらの報告\cite{babaACL2012}では一般的な英語のスペリング誤りの脱落誤りの割合が45\%前後,Babaらが抽出した英語のスペリング誤りの脱落誤りの割合が22\%前後あり,そういった結果を考慮すると本研究で抽出したスペリング誤りの結果はBabaらの結果と近い結果になったと考えられる.

以下ではキーボードのキー配置によるスペリング誤り,音韻的混同,単語内のスペリング誤りの位置に対するスペリング誤りの観点で分析を行う.

\subsection{キー配置によるスペリング誤り}
打鍵するときにキーボードのキー配置が近いことからスペリング誤りをしてしまうような例がみられた.表5.1においてスペリング誤りの文字eやrは互いにrとeによく打ち間違えている場合が該当する.他にもoとp,bとvのようなスペリング誤りに対してこの場合が原因の一つとして考えられる.

また荒牧らによる研究\cite{aramakiNLP2010}においてeとa,eとiの文字の置きかわりがよく発生することが報告されているが,表5.1や表5.2を見るとeとa,eとiのスペリング誤りの頻度はeとrの頻度と比べて少なく,キー配置が原因で起こると考えられるeとrのスペリング誤りの方が今回の結果では多く見られた.これは今回の研究ではユーザ自身が修正してしまうようなBabaらが抽出したスペリング誤り\cite{babaACL2012}も抽出していることが要因として考えられ,このことはタイピングゲームを利用することによる仮定に沿ったことである.

\begin{comment}
表5.1においてのスペリング誤りの文字と入力すべき文字をスペリング誤りのペアとして抽出し,その中で
\end{comment}
また隣り合ったキー同士のスペリング誤りのペアの割合は43.3\%になったことからも,キー配置による誤りが多いと考えられ,
この結果もタイピングゲームを利用することによる仮定に沿った結果である.

\begin{comment}
\subsubsection{視覚的混同}
文字同士の形が似ていることから視覚的混同を起こすことが原因であると考えられるスペリング誤りを示す.文字同士の形態的類似度は荒牧ら\cite{aramakiNLP2010}が用いた式に基づいて算出した.式は文字をMSゴシックのフォントで表し,2つの文字の面積の和を分母,2つの文字のフォントを重ね合わせたとき,重なり合った部分の面積の2倍の値を分子としている.式は以下に示す.

\[
  \mbox{文字の類似度} = \frac{\mbox{文字同士の重なり合う部分の面積} \times 2}{\mbox{2つの文字の面積の和}}
\]

表5.3にこの式を用いて求めたsやoと他の文字との類似度を示す.表5.1をみるとsがスペリング誤りになるときの入力すべき文字はcであり,これらの文字は表5.3で示される文字の類似度においても高い値を示している.また表5.3にはpとoについての類似度も示しており,pとoの類似度が互いに高い類似度の値を示している.このことからcとs,pとoのスペリング誤りにおいて視覚的混同が理由の一つになると考えられる.

表5.1においてのスペリング誤りの文字と入力すべき文字をスペリング誤りのペアとして抽出し,その中で表5.3の類似度が0.75以上となるスペリング誤りを抽出したところ,その割合は18.2\%となった.
\end{comment}

\subsection{音韻的混同}
表5.1でbとvは相互にスペリング誤りを起こす頻度が高い文字のペアとなっている.b と v は打鍵誤りが原因のスペリング誤りの一つであるが,これは日本人特有の音韻的混同が原因とも考えられる.rとlの置きかわりも日本人特有の音韻的混同が原因だと考えられるが,今回の結果ではrとlのスペリング誤りはスペリング誤りの文字がrで入力すべき文字がlのときの頻度が6件,スペリング誤りの文字がlで入力すべき文字がrのときの頻度が0件であったので,タイピングゲームにおけるスペリング誤りでは音韻的混同よりキー配置が原因であると考えられる.この結果はタイピングゲームを利用することによる仮定に沿った結果である.

\begin{comment}
表5.1においてのスペリング誤りの文字がbで入力すべき文字がvのときの頻度が9件,スペリング誤りの文字がvで入力すべき文字がbのときの頻度が12件であった.
\end{comment}

 \begin{table}[t]
  \small
  \begin{center}
   \caption{eとiのスペリング誤りの文字と頻度}
   \begin{tabular}{|c|c|c|} \hline
       	スペリング誤りの文字 & 入力すべき文字 & 頻度\\ \hline
	    e & i & 10\\ \hline
	    e & a & 11\\ \hline
	    a & e & 10\\ \hline
	    i & e & 4\\ \hline
   \end{tabular}
  \end{center}
 \end{table}

{\tabcolsep = 0.8mm
 \begin{table}[t]
  \small
  \centering
   \caption{単語内のスペリング誤りの位置の割合}
   \begin{tabular}{|c|c|c|c|c|c|c|c|c|c|c|} \hline
       	長さ & 頻度 & 1文字 & 2文字 & 3文字 & 4文字 & 5文字 & 6文字 & 7文字 & 8文字 & 9文字\\ \hline
	    3文字 & 49 & 0.41 & 0.14 & 0.45 &  &  &  &  &  & \\ \hline
	    4文字 & 167 & 0.246 & 0.210 & 0.275 & 0.269 &  &  &  &  & \\ \hline
	    5文字 & 196 & 0.194 & 0.173 & 0.179 & 0.276 & 0.179 &  &  &  & \\ \hline
	    6文字 & 149 & 0.128 & 0.087 & 0.195 & 0.248 & 0.195 & 0.148 &  &  & \\ \hline
	    7文字 & 120 & 0.142 & 0.083 & 0.150 & 0.200 & 0.133 & 0.142 & 0.150 &  & \\ \hline
	    8文字 & 76 & 0.04 & 0.09 & 0.11 & 0.16 & 0.11 & 0.18 & 0.22 & 0.09 & \\ \hline
	    9文字 & 58 & 0.05 & 0.10 & 0.16 & 0.09 & 0.12 & 0.05 & 0.12 & 0.22 & 0.09 \\ \hline
   \end{tabular}
 \end{table}
}

\subsection{単語内のスペリング誤りの位置}
タイピングゲームにおいて表示される英単語の文字に対して,どの位置にある文字に対してスペリング誤りが起きるかに関して分析を行う.
この観点における分析は荒牧ら\cite{aramakiNLP2010}やBabaら\cite{babaACL2012}も行っており,荒牧らは単語の語頭や語末でのスペリング誤りの頻度は語の中頃より少なくなっていることを報告した.
またBabaらは語頭での文字の脱落誤りや語末での文字の過剰誤りはユーザが気がつきやすく修正されるが,語の中頃での文字の脱落誤りや挿入誤りは気づきにくく,スペリング誤りとして残る傾向があることを報告した.

表5.3をみると語の中頃のスペリング誤りの割合が明確に多いとは言えない.これは本研究で抽出したスペリング誤りにはBabaらが抽出したユーザ自身が修正してしまうようなスペリング誤り\cite{babaACL2012}も含まれていることが原因であると考えられる.
この分析においてタイピングゲームを利用することによる仮定に沿った結果であるかどうか明確にはわからないが,少なくとも荒牧らが抽出したスペリング誤り\cite{aramakiNLP2010}と同じような結果ではないことがわかる.

\begin{comment}
表5.4には英単語の長さに応じた単語内のスペリング誤りの位置の割合を示している.表5.4から英単語の長さが長くなるに応じて語頭と語尾の文字に対するスペリング誤りの割合が減っていることがわかる.
\end{comment}

\begin{comment}
\subsubsection{スペリング誤りの割合}
表7にそれぞれの観点での分析におけるスペリング誤りの割合を示す.この結果からタイピングゲームのような通常より素早くタイピングを行ったり,文字を書き写すような状況ではキーボードのキー配置が近いことが原因で起きる打鍵誤りや入力すべき文字を飛ばしてしまう誤りが起きることがわかった.また単語の語頭や語末でのスペリング誤りの割合は表7に示す.
\end{comment}

\begin{comment}
\subsubsection{同じ文字が連続している文字列に対するスペリング誤り}
\end{comment}

\begin{comment}
\subsubsection{アルファベットに対する数字のスペリング誤り}
表5.1にはその事例を示していて,ユーザがタイピングゲームにおいて英単語ideaを入力するときにユーザが入力した文字列が13ideaであった場合があった.これはそれぞれiと1,dと3の視覚的混同によって起きたものであると考えられる.しかし英単語を入力するという設定のタイピングゲームにおいて,数字を入力することは考えにくい.しかしタイピングゲームではハイスコアを競う設定がされているので,競うようなユーザが焦るような状況においては,視覚的混同のようなスペリング誤りを引き起こす要因が高まると考えられる.
\end{comment}

\begin{comment}
表5.3と表5.4は上位10件の置換前の文字と置換後の文字のペアを示している.表5.3と表5.4を比較すると,表5.3においてxとc,vとc,jとhなど隣接するキー配置による打鍵誤りからもたらされるスペリング誤りが10件中8件見られ,表5.4ではzとs,xとsのような打鍵誤りからもたらされるスペリング誤りが10件中4件見られタイピングゲームを用いたスペリング誤りと通常のスペリング誤りで違いが見られた.表5.3においてbとvの置換もよく見られ,このスペリング誤りの原因は打鍵誤りによるものと日本人特有のローマ字の使用による影響と考えられるが,同じくローマ字の使用による影響によって起きるrとlの置換は表5.1に示すように本研究においてはあまり起こらなかったため,タイピングゲームのような表示された文字を書き写す状況ではローマ字の使用による影響をあまり受けず,打鍵誤りにより引き起こされる置換が通常のスペリング誤りに比べよく起きることが示唆された.
\end{comment}

\section{誤りに関する定性的な分析}

以下では単語同士の見間違い,同じ文字が連続している文字列に対するスペリング誤りの観点で分析を行う.この節でのスペリング誤りは目視で確認した.

\subsection{単語の見間違い}
ユーザが綴りの似ている単語同士を見間違えていると考えられるスペリング誤りが存在した.表5.4で文字列に対応した英単語はthoughだが,一方でユーザが入力した文字列はth\underline{r}oughとなっている.
\begin{comment}
これは本研究のようにユーザが入力しようとしている文字列がわからなければ抽出できない事例である.
またWheelerの研究\cite{wheeler1970processes}において人間は単語を文字や音素のような要素を逐次処理することで認識するのではなく,分けられない単体として認識している可能性があることが指摘されており,この事例はそれに伴った結果であると考えられる.
\end{comment}

また綴りが似ているだけでなく,発音が単語の見間違いに影響していると考えられるスペリング誤りが抽出された.表5.4でそのような例を示しており,文字列に対応した英単語はhereだが,一方でユーザが入力した文字列はhe\underline{a}r\underline{h}eとなっていて,これはユーザが英単語hereに対して英単語hearを入力したのではないかと考えられる.またhereとhearは発音的には同じなので,タイピングゲームにおいてhereを入力するという状況にも関わらず,英単語同士の発音の近さが影響してこういった誤りが起こったのではないかと考えられる.

 \begin{table*}[!t]
  \small
  \begin{center}
   \caption{単語同士を見間違えた例と同じスペリング誤りを繰り返した場合の例}
   \begin{tabular}{|c|c|} \hline
       	ユーザが入力した文字列 & 文字列に対応した英単語\\ \hline
	    th\underline{r}ough & though\\ \hline
	    he\underline{a}r\underline{h}e & here\\ \hline
	    sti\underline{gg}ff & stiff\\ \hline
	    s\underline{e}w\underline{wt}eet & sweet\\ \hline
   \end{tabular}
  \end{center}
 \end{table*}

\subsection{同じスペリング誤りが連続している場合}
タイピングゲームにおいて同じ文字が連続しているような英単語が表示されたとき,その文字列に対して同じスペリング誤りを繰り返してしまうといった事例が見られた.表5.4では英単語stiffに対してユーザが文字列sti\underline{gg}ffを入力した場合を示しており,これは文字fに対して文字gを入力してしまっている.このスペリング誤りはキーボードの配置による打鍵誤りによるものだと考えられ,同じ文字が連続している文字列に対してスペリングを誤った場合,同様のスペリング誤りを繰り返してしまう傾向にあると考えられる.

また文字と文字列のアルファベットが入れ替わる場合がある.表5.4では英単語sweetに対してユーザが文字列s\underline{e}w\underline{wt}eetを入力していて,この場合ユーザは文字wを入れようとしているときに文字eを,文字列eeを入れようとしているときに文字列wwを入力してしまっていると考えられる.この事例から文字と文字同士だけでなく,文字列と文字列同士のアルファベットの入れ替わりが起こりうることを示している.
