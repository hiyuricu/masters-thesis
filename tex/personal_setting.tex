\newcommand{\jdoctitle}{修士論文}
\newcommand{\edoctitle}{Master's Thesis}
\newcommand{\studentnumber}{12890526}  % 学籍番号
\newcommand{\jtitle}{英単語タイピングゲームによるスペリング誤りの抽出と分析}  % 修論の題名
\newcommand{\etitle}{Lorem Ipsum}     % 英語の題名
\newcommand{\jauthor}{立花 竜一}      % 著者名
\newcommand{\eauthor}{Ryuichi Tachibana} % 英語の著者名
\newcommand{\jdate}{\today}
\newcommand{\edate}{\ifcase\month\or
    January\or February\or March\or April\or May\or June\or
    July\or August\or September\or October\or November\or December\fi
    \space\number\day,\space \number\year}
\newcommand{\keywords}{修士論文,首都大学東京大学院}
\newcommand{\ekeywords}{Master's Thesis, TMU}

\newcommand{\jabstract}{ % 日本語概要
これまで英語のスペリング誤り抽出に関する多くの研究が行われてきた.それらの研究ではTwitterなどのwebサービスからスペリング誤りの候補を抽出する,クラウドソーシングを利用した入力ログからスペリング誤りの候補を得るといった方法でスペリング誤りを収集していた.しかし,そういった研究では抽出したスペリング誤りが何の単語のスペリング誤りかわからない,クラウドソーシングのコストがかかるといった難点があった.

そこで本研究では英単語タイピングゲームを利用することで,スペリング誤りに対応する単語が明らかであり,クラウドソーシングのコストもかからないスペリング誤りの抽出手法を提案し,実際に抽出したスペリング誤りに関する分析を行う.
}

\newcommand{\eabstract}{ % 英語概要
 aaaaaaaaaaaaaaaaaaa
}