\chapter{はじめに}
パソコンは主にキーボードを利用して操作を行うが,キーボードを利用する場合,スペリングを誤ってしまうことがある.たとえば,隣接するキーを誤って押してしまう混同や,アルファベット同士の音韻的・視覚的類似性によって間違えてしまう誤りがある.本研究ではキーボードの誤操作によって発生するタイポと誤認識によって発生する綴りの混同を合わせて\textbf{スペリング誤り}と定義する.

パソコンの普及率の向上に伴い,スペリング誤りの検出・訂正が注目され,スペリング誤りを分析してその原因を解明しようとする研究やスペリング誤り訂正に貢献する研究が行われている.たとえば,Twitterなどのwebサービスからスペリング誤りの候補を抽出する\cite{aramakiNLP2010},クラウドソーシングを利用した入力ログからスペリング誤りの候補を得る\cite{babaACL2012}などしてスペリング誤りの獲得を行う研究がある.
しかしコーパスから教師なしにスペリング誤りの候補を抽出する場合何の単語のスペリング誤りかわからない,クラウドソーシングを利用する場合コストがかかるといった難点があった.

一方ゲームにおいて用いられている要素をタスクに活用することによって,ユーザがタスクにより取り組んでもらえるようにタスクの設定を工夫するゲーミフィケーションの研究が盛んになってきている\cite{deterdingACM2011}.教師なしに知識獲得をするのと異なり,ゲーミフィケーションでは自分でタスクを設定することができる利点がある.また,ゲーミフィケーションではユーザに対価を払うことなく,何らかのタスクにおいてのリソースを獲得できるという利点がある.

そこで本研究ではタイピングゲームを利用することで,スペリング誤りに対応した単語は何か分かっており,コストもかからないスペリング誤りの抽出手法を提案する.抽出したスペリング誤りに関する分析を行い,スペリング誤り訂正に貢献することを目的とする.タイピングゲームを利用してスペリング誤り訂正に貢献することが,一般に存在するタイピングゲームと異なっている.

本研究ではタイピングゲームを利用することから以下のような仮説を立てた.

\begin{itemize}
 \item 本研究のタイピングゲームでは速くタイピングゲームをクリアするほどより高いスコアを獲得することができ,ユーザは急いで文字を入力するため,打鍵ミスによるスペリング誤りの割合が大きくなる.
 \item ユーザは入力する文字を見て入力しているため,音韻的混同が原因によるスペリング誤りの頻度が相対的に小さくなる.
 \item Babaらが抽出した最終出力文字列に残らない,ユーザが文字を入力中に修正するスペリング誤り\cite{babaACL2012}と最終出力文字列に残る一般的なスペリング誤りの両方を抽出するため,それぞれの誤りが混合した結果になる.
\end{itemize}

これらの仮説がスペリング誤りに関する定量的な分析において正しいかどうか抽出された結果に基づいて考察する.

スペリング誤りを抽出するためにタイピングゲームを利用することは,抽出されるスペリング誤りに影響を与えるような状況を生み出すと考えられ,その状況は以下である.

\begin{itemize}
 \item ユーザは入力する文字を見て入力する状況になる.ユーザが入力する文字がわかるという利点があるが,入力する文字を見て入力する状況が影響したスペリング誤りの抽出になる.
 \item タイピングゲームにおいてユーザが急いで文字を入力する状況になる.速く文字を入力することで平常時より多くのスペリング誤りを抽出できる可能性があるが,急いで文字を入力する状況が影響したスペリング誤りの抽出になる.
 \item 本研究で利用したタイピングゲームでは既存のタイピングゲーム\cite{game}と同様にユーザ自身が文字を消去する設計がない.そのため本研究での手法が既存のタイピングゲームへの応用に繋がる可能性があるが,文字の過剰誤りと置換誤りを区別することや,抽出したスペリング誤りがBabaらが抽出した最終出力文字列に残らない,ユーザが文字を入力中に修正するスペリング誤り\cite{babaACL2012}かどうか判断することができない.
\end{itemize}

本研究では荒牧ら\cite{aramakiNLP2010}やBabaら\cite{babaACL2012}の実験における状況と比べてこのような難点があるが,我々が抽出したスペリング誤りの結果とBabaらが抽出したスペリング誤りの結果を比較して,それぞれの結果が似たような結果であることがわかれば,タイピングゲームを利用して抽出したスペリング誤りがスペリング誤り訂正に貢献するために活用できると考えられる.

本研究の主要な貢献は以下である.

\begin{itemize}
 \item 我々の知る限り,タイピングゲームをスペリング誤り抽出に用いた研究は本研究が初めてである.
 \item ユーザが入力しようとしている英単語がわかっているため,編集距離に基づくスペリング誤り抽出手法\cite{aramakiNLP2010}では行えないようなスペリング誤りに関しての分析が行える.
 \item スペリング誤り抽出のためにタイピングゲームを利用する状況での仮説が正しいことを示す.
 \item タイピングゲームを利用して抽出したスペリング誤りがスペリング誤り訂正に貢献するために活用できることを示す.
\end{itemize}
