\chapter{はじめに}
パソコンは主にキーボードを利用して操作を行うが,キーボードを利用する場合,スペリングを誤ってしまうことがある.たとえば,隣接するキーを誤って押してしまう混同や,アルファベット同士の音韻的・視覚的類似性によって間違えてしまう誤りがある.本研究ではキーボードの誤操作によって発生するタイポと誤認識によって発生する綴りの混同を合わせて\textbf{スペリング誤り}と定義する.

パソコンの普及率の向上に伴い,スペリング誤りの検出・訂正が注目され,スペリング誤りを分析してその原因を解明しようとする研究やスペリング誤り訂正に貢献する研究が行われている.たとえば,Twitterなどのwebサービスからスペリング誤りの候補を抽出する\cite{aramakiNLP2010},クラウドソーシングを利用した入力ログからスペリング誤りの候補を得る\cite{babaACL2012}などしてスペリング誤りの獲得を行う研究がある.
しかしコーパスから教師なしにスペリング誤りの候補を抽出する場合何の単語のスペリング誤りかわからない,クラウドソーシングを利用する場合コストがかかるといった難点があった.

一方ゲームにおいて用いられている要素をタスクに活用することによって,ユーザがタスクにより取り組んでもらえるようにタスクの設定を工夫するゲーミフィケーションの研究が盛んになってきている\cite{deterdingACM2011}.教師なしに知識獲得をするのと異なり,ゲーミフィケーションでは自分でタスクを設定することができる利点がある.また,ゲーミフィケーションではユーザに対価を払うことなく,何らかのタスクにおいてのリソースを獲得できるという利点がある.

そこで本研究ではタイピングゲームを利用することで,スペリング誤りに対応した単語は何か分かっており,コストもかからないスペリング誤りの抽出手法を提案する.抽出したスペリング誤りに関する分析を行い,スペリング誤り訂正に貢献することを目的とする.タイピングゲームを利用してスペリング誤り訂正に寄与することが,一般に存在するタイピングゲームと異なっている.

本研究のようなタイピングゲームを利用する状況では以下のことを仮定する.

\begin{itemize}
 \item 本研究のタイピングゲームでは速くタイピングゲームをクリアするほどより高いスコアを獲得することができ,ユーザは急いで文字を入力するため,打鍵ミスによるスペリング誤りの割合が大きくなる.
 \item ユーザは入力する文字を見て入力しているため,音韻的混同が原因によるスペリング誤りの頻度が相対的に小さくなる.
 \item 荒牧らが抽出したスペリング誤り\cite{aramakiNLP2010}とBabaらが抽出したスペリング誤り\cite{babaACL2012}の両方を抽出するため,それぞれの結果が混合した結果になる.
\end{itemize}

これらの仮定がスペリング誤りに関する定量的な分析において正しいかどうか抽出された結果に基づいて考察する.

また本研究の主要な貢献は以下である.

\begin{itemize}
 \item 我々の知る限り,タイピングゲームをスペリング誤り抽出に用いた研究は本研究が初めてである.
 \item ユーザが入力しようとしている英単語がわかっているため,編集距離に基づくスペリング誤り抽出手法\cite{aramakiNLP2010}では行えないようなスペリング誤りに関しての分析が行える.
 \item スペリング誤り抽出のためにタイピングゲームを利用する状況での仮定が正しいことを示す.
\end{itemize}
