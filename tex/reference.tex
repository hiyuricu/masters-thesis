 \begin{thebibliography}{99}
	\bibitem[1]{aramakiNLP2010} 荒牧英治,宇野良子,岡 瑞起:TYPO Writer:ヒトはどのように打ち間違えるのか?,言語処理学会第 16 回年次大会予稿集, pp. 966-969 (2010).
	
	\bibitem[2]{babaACL2012} Baba, Y. and Suzuki, H.: How are spelling errors generated and corrected? A study of corrected and uncorrected spelling errors using keystroke logs, Proceedings of the 50th Annual Meeting of the Association for Computational Linguistics (Volume 2: Short Papers), pp. 373-377 (2012).

	\bibitem[3]{game} インターネットでタイピング練習 e-typing: \url{http://www.e-typing.ne.jp/english/}.

	\bibitem[4]{deterdingACM2011} Deterding, S., Dixon, D., Khaled, R. and Nacke, L.:From game design elements to gamefulness: Defining “”Gamification“”, Proceedings of the 15th International Academic MindTrek Conference: Envisioning Future Media Environments, MindTrek ’11, pp. 9-15 (2011).

	\bibitem[5]{kernighan1990spelling} Kernighan, M. D., Church, K. W. and Gale, W. A.: A spelling correction program based on a noisy channel model, Proceedings of the 13th Conference on Computational Linguistics-Volume 2, pp. 205-210 (1990).

	\bibitem[6]{brill2000improved} Brill, E. and Moore, R. C.: An improved error model for noisy channel spelling correction, Proceedings of the 38th Annual Meeting of the Association for Computational Linguistics, pp. 286-293 (2000).

	\bibitem[7]{ahmad2005learning} Ahmad, F. and Kondrak, G.: Learning a spelling error model from search query logs, Proceedings of the Conference on Human Language Technology and Empirical Methods in Natural Language Processing, pp. 955-962 (2005).

	\bibitem[8]{cook1997l2} Cook, V. J.: L2 users and English spelling, Journal of Multilingual and Multicultural Development, Vol. 18, No. 6, pp. 474-488 (1997).

	\bibitem[9]{kumaranCOLING2014} Kumaran, A., Densmore, M. and Kumar, S.: Online gaming for crowd-sourcing phrase-equivalents, Proceedings of COLING 2014, the 25th International Conference on Computational Linguistics: Technical Papers, pp. 1238-1247 (2014).

	\bibitem[10]{vannella2014validating} Vannella, D., Jurgens, D., Scarfini, D., Toscani, D. and Navigli, R.: Validating and extending semantic knowledge bases using video games with a purpose, Proceedings of the 52nd Annual Meeting of the Association for Computational Linguistics, pp. 1294-1304 (2014).

	\bibitem[11]{venhuizen2013gamification} Venhuizen, N., Basile, V., Evang, K. and Bos, J.: Gamification for word sense labeling, Proceedings of the 10th International Conference on Computational Semantics (IWCS-2013), pp. 397-403 (2013).

	\bibitem[12]{simpleenglish} Ogden, C. K.: Basic English: A general introduction with rules and grammar, London: Paul Treber \& Co., Ltd. (1930).


 \end{thebibliography}


 \newpage

 \section*{\Large 発表リスト}
 \addcontentsline{toc}{chapter}{発表リスト}
 \noindent [NL222] 立花竜一, 小町守:英単語タイピングゲームによるスペリング誤りの抽出と分析, 研究報告自然言語処理(NL), 2015-NL-222(10), 1-7 (2015-07-08).