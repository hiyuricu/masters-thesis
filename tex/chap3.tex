\chapter{タイピングゲームによるスペリング誤り抽出}\chaplab{result}
本研究ではユーザがタイピングゲーム\footnote{\url{http://cl.sd.tmu.ac.jp/~ryu/typing_game.html}}を行うことで,ユーザがタイプしたアルファベットの文字列を抽出し,その文字列に対応した英単語と比較を行うことでスペリング誤りの獲得を行う.ユーザが入力する文字列にはスペリング誤りが含まれているため,ユーザがタイプしたアルファベットの文字列と英単語を比較することでスペリング誤りの分析が可能になる.

ユーザが入力したアルファベットの文字列とその文字列に対応した英単語の比較は,タイピングゲームにおいて間違いだと判定される文字を抽出するように比較が行われる.例えば表1ではユーザが入力した文字列とその文字列に対応した英単語の比較例を示している.入力すべき英単語はbeliefだったが,ユーザが実際に入力した文字列はbel\underline{e}iefであったとき,beliefの4文字目にあたるiを入力すべきときにアルファベットeを入力しているので,アルファベットeをスペリング誤りとして抽出を行う.スペリング誤りのアルファベットeと,ユーザが入力しようとしていたbeliefの4文字目にあたるiと,その前後の文字をスペリング誤りの原因の分析のためにそれぞれ抽出する.

ユーザが入力した文字列において複数のスペリング誤りが存在し,それらがタイピングゲームにおいて正解だと判定される文字または文字列で区切られている場合はそれぞれのスペリング誤りは別々のスペリング誤りとして扱うものとする.表2では1つの文字列において複数のスペリング誤りが抽出される例を示しており,ユーザが入力した文字列in\underline{g}cre\underline{s}aseにおいて3文字目のgと7文字目のsをスペリング誤りとして抽出している.3文字目のgをスペリング誤りとして扱ったあと,ingcresaseにおいて4文字目から6文字目creは文字列に対応した英単語increaseの3文字目から5文字目と対応して,タイピングゲームにおいて正解だと判定される.そのあとユーザは英単語increaseの6文字目の文字aを入力する必要があるが,ユーザは文字列ingcresaseの7文字目のsを入力しているので,結果gとsがスペリング誤りとして抽出される.

タイピングゲームにおいてユーザがスペリング誤りだと判定される文字を連続で入力する場合もあるが,その場合連続した文字列がタイピングゲームにおいて入力すべき文字以降の英単語の部分文字列であれば,その文字列はスペリング誤りとは判定せずに分析を行う.表2にはスペリング誤りと判定されない場合の文字や文字列の例が示されており,ユーザが入力した文字列は\underline{inly}onlyで,文字列に対応した英単語はonlyなので,タイピングゲームにおいて間違いだと判定される文字はそれぞれiとnとlとyであるが,inlyにおいて文字列nlyはそれぞれ入力すべき文字o以降の部分文字列なので,文字列nlyはそれぞれスペリング誤りとして抽出をしないものとする.

\begin{comment}
タイピングゲームにおいてスペリング誤りだと判定される文字が連続で入力され,それらの文字の連続が全てスペリング誤りだと判定された場合,それらの文字の連続は文字列として分析を行う.
\end{comment}
また表2にはスペリング誤りと判定される文字列の例が示されており,ユーザが入力した文字列bi\underline{yy}tから
\begin{comment}
抽出されるスペリング誤りの文字列としてyyを分析する.
このような編集距離が2以上の文字列に関してのスペリング誤りの確認は目視で行った.
抽出した文字列に含まれる文字は文字単位のスペリング誤りとして扱うことができるので,1文字ずつに分けて文字としても分析を行うものとし,表2のユーザが入力した文字列がbi\underline{yy}tの場合では,
\end{comment}
抽出されるスペリング誤りの文字としてyが2文字抽出され,入力すべき文字はtとする.

  \begin{table*}[t]
  \small
%  \begin{center}
   \caption{ユーザが入力した文字列と文字列に対応した英単語の比較とスペリング誤りの抽出例}
   \begin{tabular}{|c|c|c|c|c|c|} \hline
       	入力した文字列 & 対応した英単語 & スペリング誤り & 1つ前の文字 & 入力すべき文字 & 1つ後の文字\\ \hline
	    bel\underline{e}ief & belief & e & l & i & e\\ \hline
   \end{tabular}
%  \end{center}
 \end{table*}

 \begin{table*}[t]
  \small
  \begin{center}
   \caption{スペリング誤りと判定される文字列,されない文字列と複数のスペリング誤りが抽出される文字列の例}
   \begin{tabular}{|c|c|c|c|} \hline
       	ユーザが入力した文字列 & 文字列に対応した英単語 & スペリング誤りの文字 & スペリング誤りの文字列\\ \hline
		in\underline{g}cre\underline{s}ase & increase & gとs & \\ \hline
		\underline{inly}only & only & i & \\ \hline
	    bi\underline{yy}t & bit & yとy & yy\\ \hline
   \end{tabular}
  \end{center}
 \end{table*}
