\chapter{関連研究}\chaplab{definition}
まずスペリング誤りに関しての関連研究について説明する.
KernighanらはNoisy Channel Modelを使って単語が与えられたときの訂正候補の確率をモデル化することでスペリング訂正を行った\cite{kernighan1990spelling}.この研究ではスペリング訂正モデルは1文字同士の挿入,削除,置換,転置といった編集距離の値に基づいて計算された.またBrillらはKernighanらと同様にNoisy Channel Modelによるスペリング誤りの訂正を行ったが,訂正候補の確率をモデル化するときに一文字だけでなく,文字列の編集距離を考慮することでスペリング誤りの訂正の精度を向上させた\cite{brill2000improved}.
Ahmadらはウェブ検索クエリログからスペリングを訂正するためのモデルを自動的に学習した\cite{ahmad2005learning}.この研究ではEMアルゴリズムを手法とすることで,正解データを必要とせずに重み付き編集距離を学習した.
重み付き編集距離はaとeは誤りやすいが,aとlは誤りにくいといった文字同士の誤りやすさに関する情報を反映するために利用される.
\begin{comment}
Ahmadらはまず重み付けされていないモデルを用いて誤りを検出し,その誤りを用いてモデルの重み付けを更新することで学習を行った.
\end{comment}
これらの研究ではスペリング誤りの訂正のために研究を行っているが,本研究ではスペリング誤りの訂正は行わず,スペリング誤りの特徴の分析を行っている.

Cookによる研究\cite{cook1997l2}では英語を母語とする第一言語話者と母語としない第二言語話者の英語のスペリング誤りを比較して分析を行っている.本研究では英語を母語としないユーザのスペリング誤りを分析している点が共通しているが,Cookはスペリング誤りのデータを学生がテストや宿題で書いたものなどから抽出していたのに対し,本研究ではタイピングゲームを用いてスペリング誤りを抽出する.

荒牧らはTwitterのクロールデータを利用することで英単語のスペリング誤りの抽出を行い,スペリング誤りの原因を分析した\cite{aramakiNLP2010}.またスペリング誤りとスペリング誤りでないものを判別する学習器を構築し,実験を行うことで分析結果の検証を行った.クロールしたデータにおいてスペリング誤り候補を決定するために,英単語から編集距離が1であるものを収集した.
本研究における荒牧らの研究との共通点は,キー配置によるスペリング誤りや単語内のスペリング誤りの位置といった観点でスペリング誤りの分析を行うことである.また荒牧らの研究では正解の単語を推測するために編集距離を利用していたが,本研究では正解の単語を推測する必要がない手法を提案する.

Babaらはクラウドソーシングの1つであるAmazon Mechanical Turkを利用して,ある画像が何を描写しているかユーザに英文を入力させユーザのキーストロークを抽出し,スペリング誤りに対応する単語が分かるスペリング誤り候補を抽出する手法を実装し,その結果に対して分析を行った\cite{babaACL2012}.修正前文字列と修正後文字列の編集距離が2以下であるものを抽出し,それらを比較してスペリング誤り候補を抽出することで分析を行った.本研究とはスペリング誤りに対応する単語がわかっている点が同じだが,この研究ではAmazon Mechanical Turkを利用するためコストがかかるのに対し,本研究ではコストのかからないタイピングゲームを用いたスペリング誤りの抽出手法を提案する.

またユーザがゲームを行うことを通して言語資源を得るという研究がある.Kumaranらはある句に対して同じ意味の句を得るためにユーザにゲームを行わさせた\cite{kumaranCOLING2014}.そのゲームはあるユーザがある句に関しての絵を描き,その絵を他のユーザに見せてその絵が何の句を示しているのか答えてもらうというものであった.
Vannellaらは意味的な知識の検証と拡張のために2つのゲームを作成し,ユーザに行わさせた\cite{vannella2014validating}.2つのゲームはそれぞれシューティングゲームとロールプレイングゲームを模したものであり,それらのゲームを行うことで概念同士の,または概念と画像の関係のアノテーションを行わせた.ユーザがゲームを通して行ったアノテーションは,クラウドソーシングにおいて労働者が報酬を受け取って作成したアノテーションと比較しても高い品質のものであった.
VenhuizenらはWordrobeと呼ばれるゲームをユーザに行わせることで語義のラベル付けを行った\cite{venhuizen2013gamification}.ゲームは語義に関する複数の選択がある質問の集合で構成されていて,複数のユーザはそれに答えて,他のユーザと答えが一致しているかどうかによってユーザが獲得する得点が決まる.
\begin{comment}
得られたデータの量は比較的少なかったが,適合率と再現率の結果は期待できるものであるとしている.
\end{comment}
これらの研究と本研究との共通点はゲームを用いて言語知識獲得を行う点であるが,相違点は得ることができる言語資源がスペリング誤りである点である.