% -*- coding: cp932-dos -*-
\documentclass{jarticle}

\usepackage{icspresen_m22}

\No{A-00}
\Jtitle{発表会予稿の書き方に関する研究}
\Etitle{A Study of How to Prepare the Manuscript}
\Author{00000000___首都_太郎_______________指導教官__首都_花子_教授}
\presenyear{26}
\presendate{平成27年2月6日}
\master   % 修士論文発表会の場合
%\bachelor % 特別研究発表会の場合


\begin{document}
\mktitle
%
\renewcommand{\baselinestretch}{1.1}\small
%
\section{はじめに}
この文章は,{\tt icspresen.sty} の使用法を説明しています.
以下の説明を読み,フォーマットに関しては間違いのない予稿を作成してください.

\section{設定が必要なマクロ}

本文の前に以下のマクロの記述をする必要があります.
\newcommand{\tbs}{\textbackslash}
\newcommand{\macrocom}[1]{{\tt \textbackslash#1}}

\begin{quote}
\macrocom{No}\{A-00\}\\
\macrocom{Jtitle}\{発表会予稿の書き方に関する研究\}\\
\macrocom{Etitle}\{A Study of How to Prepare the Manuscript\}\\
\macrocom{Author}\{00000000\_\_\_首都\_太郎\_\_\_指導教官\_\_首都\_花子\_教授\}\\
\macrocom{presenyear}\{23\}\\
\macrocom{presendate}\{平成24年2月8日\}\\
\macrocom{master}   \% 修士論文発表会の場合\\
\macrocom{bachelor} \% 特別研究発表会の場合
\end{quote}

\macrocom{No} は発表番号を設定します.
\macrocom{Jtitle} および \macrocom{Etitle} は日本語と英語の
論文名をそれぞれ設定します.本文を英語で書く場合でも両方を設定
してください.

\macrocom{Author} は氏名および指導教官を設定します.学習番号,氏名,指導教官の順で
記述します.また `\_' はスペースを表しますので,必要な数を挿入し調整します.

\macrocom{presenyear} は卒業・修了の年度を数字で設定します.また,\macrocom{presendate} 
は発表会の日程を設定します.発表年度と日付の年と間違えない様に注意すること.

最後の \macrocom{master} および \macrocom{bachelor} は,どちらか一方のみを
残します.修士論文発表会の場合は \macrocom{master} を,特別研究発表会の場合は,
\macrocom{bachelor} を残します.間違えるとフッターの情報が間違ったものとなって
しまうので,注意すること.

\section{本文について}
ここでは,本文の記述について説明しています.本文は2ページまで記述することができます(修士論文発表会予稿の場合は4ページも可).
また,上下左右のマージンに関しては,このスタイルファイルのものから変更しないで下さい.

\subsection{文字サイズなど}
本文の文字サイズなどは特に指定はありません.長さや読みやすさなどを考慮し,適宜調整してください.
\subsection{セクションなど}
節などは,\LaTeX のコマンドの \macrocom{section} や \macrocom{subsection} を通常の論文と
同様に用いて下さい.

\subsection{図表など}
図や表に関しても特にフォーマットは定めていません.

\subsection{参考文献}
参考文献も通常の \LaTeX のコマンドを使用してください.これは,使用例です\cite{tarou1}.
\section{おわりに}
不明な点は問い合わせを行って下さい.


\begin{thebibliography}{9}\footnotesize
\bibitem{tarou1} 首都太郎, ``特別研究'' 首都大学東京システムデザイン学部., VOL.1, NO.1,
	 pp.1-2, 2008年12月.

\end{thebibliography}
%
\end{document}
